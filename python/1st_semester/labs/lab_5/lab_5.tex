\documentclass[a4paper]{article}
\renewcommand{\rmdefault}{ftm}
\usepackage[14pt]{extsizes}
\usepackage[utf8]{inputenc}
\usepackage[russian]{babel}
\usepackage{setspace,amsmath}
\usepackage[left=25mm, top=15mm, right=10mm, bottom=25mm, nohead, footskip=10mm]{geometry}
\begin{document}
\begin{center}
\hfill \break
\large{\textbf{ФГБОУ ВО«Московский Политехнический университет»}}\\
\hfill \break
\hfill \break
\hfill \break
\hfill \break
\hfill \break
\hfill \break
\hfill \break
\large{Лабораторная работа№5}\\
\footnotesize{Двумерные массивы и функции\\
Задание 1\hspace{3cm}Вариант№27\break\\
По дисциплине:\\
Основы Программирования
}
\end{center}
\hfill \break
\hfill \break
\hfill \break
\hfill \break
\hfill \break
\hfill \break
\hfill \break
\hfill \break
\hfill \break
\hfill \break
\normalsize{ 
\begin{tabular}{ccc}
\hspace{4cm}Выполнил & Шукуров Ф.Ф  & группа 181-362\\
\\
\hspace{4cm}Проверил & \underline{\hspace{3cm}}& Никишина И.Н
\end{tabular}
}
\hfill \break
\hfill \break
\hfill \break
\hfill \break
\hfill \break
\hfill \break
\hfill \break
\hfill \break
\hfill \break
\hfill \break
\hfill \break
\hfill \break
\begin{center}\texttt{Москва 2018}\end{center}
\thispagestyle{empty}

\newpage
Лабораторная работа№5
\\
    \begin{lab5}
        \begin{center}\underline{\hspace{6cm}}
            задание:\\
            \hspace{1cm}Даны две матрицы одного порядка M x N ( М строк х N столбцов).\\
            Написать программу сложения, вычитания и траспорнирования матриц.\\
            \begin{enumerate}
                \item Сложение и вычитание: C$_i_j$= a$_i_j\pm$b$_i_j$
                \item Транспонирование b$_i_j$=a$_j_i$
            \end{enumerate}
        \end{center}
    \begin{description}
        Описание программы:\\
        \small{Программа была написанна на python 3.6, реализованна в среде os Linux, отвечает за ввод данных, вычисление и вывод данных на экран. Был испортирован ранее установленный модуль \underline{numpy}, а так же, \underline{random, math}, для проверки пользовательского ввода был использован блок try$\to$except}
    \end{description}
    \begin{algoritm}
        Описание Алгоритма:
        \small\begin{enumerate}
            \item Импортируем все функции, создаем блок try$\to$except.
            \item с помощью метода input() а так же присваивания его к вещественноему значению создаем <<m>> и <<n>>.
            \item с помощью модуля <<numpy>> (далее <<np>>), присваиваем к двум матрицам matrix\_a и matrix\_b случайные числа в диапазоне [-100;100] размерностью указанной ранее пользователем <<n>> и <<m>>
            \item спрашиваем пользователя какой вид математических действий будем совершать над матрицами. Ответ присваиваем к переменной <<output>>
            \item Создав блок исключений (if$\to$elif$\to$else) и задав необходимые условия, выполняем те или иные действия.
            \item Для транспонирования матрицы используем функцию np$\to$transpose(matrix\_a)
            \item Вывод ответа на экран
            \item В случае ошибки метод except выводит информацию о ошибке ввода. (except $\to$ print('Ты не прав.')
        \end{enumerate}
    \end{algoritm}
        \texttt{Листинг Программы:}
    \begin{verbatim}
# -*- coding: utf-8 -*-
import numpy as np
import random, math
try:
    sum_or_subtraction = None
    m = int ( input( 'Укажите количество столбцов: \n'))
    n = int ( input( 'Укажите количество строк: \n'))
    matrix_a = np.random.randint(100, size=(n,m))
    matrix_b = np.random.randint(100, size=(n,m))
    print('Задание№1\n________________\nСложение или вычитание матриц?')
    output = str(input('\n1)+\n2)-\n'))
    if output == "1" or output == "+":
        matrix_c = matrix_a + matrix_b
        sum_or_subtraction = " \n+\n "
    elif output == "2" or output == "-":
        matrix_c = matrix_a - matrix_b
        sum_or_subtraction = " \n-\n"
    print("________________\n"+ str(matrix_a) + "\n"+ sum_or_subtraction + "\n" + str(matrix_b) + "\n=\n" + str(matrix_c) + "\n________________\nЗадание№2")
    transpose_matrix_a = np.transpose(matrix_a)
    print(str(matrix_a) + "\n\n" + str(transpose_matrix_a))
except:
    print('Ты не прав.')
    \end{verbatim}
    \begin{center}Результат программы:\end{center}
    \begin{verbatim}
Укажите количество столбцов: 
10
Укажите количество строк: 
10
Задание№1
________________
Сложение или вычитание матриц?

1)+
2)-
-
________________
[[98 13 48 44  2 39  7 76 72 41]
 [59 40 78 95 78 98 72 28 99 69]
 [42 72 90 63 16 82 91 67 66 33]
 [17 91 24 34 76 39 40 37 14 44]
 [12 31 66 53 98 71 64  0 91 11]
 [76 55  6 82 39 90 59 14  1 60]
 [ 6 91 12 36 80 81 98 22 95 98]
 [51 93 63 91 72 88 47 29 91 21]
 [76 10 52 29  1 25 10 68  6 90]
 [ 8  5 38 51 13 88 91 49 92 12]]
 
-

[[79 30 22 13 82 86 74  9 41 81]
 [23 81 80 70 26  6 27 99  6 93]
 [42  3 50 50 97 73 48 94 64 91]
 [ 4 72 77 62 16 43 67 45 67 56]
 [69 14 50 16 97 40 87 92 12  4]
 [35 55 42 54 35 29 33 86  9 29]
 [27 21 92 36 69 17 12 33 49 63]
 [ 0 47  5 94 94 25 11 29 50 24]
 [98  0 28 18 32 72 57 23 74 99]
 [32 68 20  2 41 93 64 43 30  8]]
=
[[ 19 -17  26  31 -80 -47 -67  67  31 -40]
 [ 36 -41  -2  25  52  92  45 -71  93 -24]
 [  0  69  40  13 -81   9  43 -27   2 -58]
 [ 13  19 -53 -28  60  -4 -27  -8 -53 -12]
 [-57  17  16  37   1  31 -23 -92  79   7]
 [ 41   0 -36  28   4  61  26 -72  -8  31]
 [-21  70 -80   0  11  64  86 -11  46  35]
 [ 51  46  58  -3 -22  63  36   0  41  -3]
 [-22  10  24  11 -31 -47 -47  45 -68  -9]
 [-24 -63  18  49 -28  -5  27   6  62   4]]
________________
Задание№2
[[98 13 48 44  2 39  7 76 72 41]
 [59 40 78 95 78 98 72 28 99 69]
 [42 72 90 63 16 82 91 67 66 33]
 [17 91 24 34 76 39 40 37 14 44]
 [12 31 66 53 98 71 64  0 91 11]
 [76 55  6 82 39 90 59 14  1 60]
 [ 6 91 12 36 80 81 98 22 95 98]
 [51 93 63 91 72 88 47 29 91 21]
 [76 10 52 29  1 25 10 68  6 90]
 [ 8  5 38 51 13 88 91 49 92 12]]

[[98 59 42 17 12 76  6 51 76  8]
 [13 40 72 91 31 55 91 93 10  5]
 [48 78 90 24 66  6 12 63 52 38]
 [44 95 63 34 53 82 36 91 29 51]
 [ 2 78 16 76 98 39 80 72  1 13]
 [39 98 82 39 71 90 81 88 25 88]
 [ 7 72 91 40 64 59 98 47 10 91]
 [76 28 67 37  0 14 22 29 68 49]
 [72 99 66 14 91  1 95 91  6 92]
 [41 69 33 44 11 60 98 21 90 12]]
    \end{verbatim}
    \end{lab5}
    
\end{document}
