\documentclass[a4paper]{article}
\renewcommand{\rmdefault}{ftm}
\usepackage[12pt]{extsizes}
\usepackage[utf8]{inputenc}
\usepackage[russian]{babel}
\usepackage{enumitem}
\usepackage{graphicx}
\setlist[enumerate]{itemsep=0mm}
\usepackage{setspace,amsmath}
\usepackage[left=25mm, top=15mm, right=10mm, bottom=25mm, nohead, footskip=10mm]{geometry}
\begin{document}
\begin{center}
\hfill \break
\large{\textbf{ФГБОУ ВО«Московский Политехнический университет»}}\\
\hfill \break
\hfill \break
\hfill \break
\hfill \break
\hfill \break
\hfill \break
\hfill \break
\large{Лабораторная работа№4}\\
\footnotesize{Одномерные массивы\\
Задание 1\hspace{3cm}Вариант№1\break\\
По дисциплине:\\
Основы Программирования
}
\end{center}
\hfill \break
\hfill \break
\hfill \break
\hfill \break
\hfill \break
\hfill \break
\hfill \break
\hfill \break
\hfill \break
\hfill \break
\normalsize{ 
\begin{tabular}{ccc}
\hspace{4cm}Выполнил & Шукуров Ф.Ф  & группа 181-362\\
\\
\hspace{4cm}Проверил & \underline{\hspace{3cm}}& Никишина И.Н
\end{tabular}
}
\hfill \break
\hfill \break
\hfill \break
\hfill \break
\hfill \break
\hfill \break
\hfill \break
\hfill \break
\hfill \break
\hfill \break
\hfill \break
\hfill \break
\begin{center}\texttt{Москва 2018}\end{center}
\thispagestyle{empty}

\newpage
Лабораторная работа №4
\\
    \begin{lab4}
        \begin{center}\underline{\hspace{6cm}}\\
        Задание№4.27\\
        \end{center}
        {\hspace{2mm}}С использованием модулям Random сформировать одномерный массив состоящий из n вещественных элементов в котором элементы случайным образом принимают положительный или отрицательный знак и значение от -5 до 5. Для заданного числа y, такого, что \textsc{amin< y < amax}, вычилить:\\
        1. Сумму элементов массива, значение модуля которы меньше y\\
        2. Произведение остальных элементов.\\
    \begin{description}
        Описание программы:\\
        Программа была написана на алгоритмическом языке python v3.6, реализованна в среде os Linux, и состоит из блоков ввода, проверки информации и вывода результата. Использован импорт random для вывода <<Псведо случайного числа>>, а так ипорт функции math.
    \end{description}
    \begin{algoritm}
        Описание Алгоритма:
        \small\begin{enumerate}
            \item
                Для начала создадим два пустых массива, для будущих операций над ними, а так же к переменной <<n>> будем пивязывать вещественное значение с помощью функции input() пользовательский ввод с клавиатуры. ( где <<n>> $\to$ количество элементов массива)
            \item
                С помощью цикла for i in range(n), а так же функции massive.append(\underline{random.randint(-5,5)}) в теле цикла, мы будем добавлять после каждой логической итерации цикла случайное число в диапазоне от -5 до 5 \textsc{[-5;5]}
            \item 
                Создавая бесконечный цикл \texttt{While} используя в качестве аргумента \underline{True}, присваиваем значению <<y>> с помощью функции input() пользовательское вещественное значение в диапазоне от -5 до 5.
            \item 
                С помощью блоков исключения (if $\to$ elif $\to$ else) проверяем принадлежность <<y>> к заданному выше диапазону. В случае исключения, программа попросит повторнный ввод значени <<y>>
            \item 
                В случае принадлежность числа к диапазону, следующий блок проверки исключений (if y not in massive) проверяет на принадлежность <<y>> к элементам массива. В случае <<Истины>>, цикл While прекращается, и программа продолжает работу. В случае <<Исключения>>, программа удаляет случайный элемент массива с помощью функции massive.remove(\underline{\hspace{1mm}}), где в качества аргумента передаем случайный элемент \textbf{massive.remive(massive[random.rangrange(n)])}, после чего добавляем к массиву значение <<y>> и выходим из цикла
            \item 
                Добавляем каждый элемент <<massive>> проходя через функцию math.fabs(), к <<massive\_absol>>, получая их модульное значение.
            \item
                Инициализируя новый цикл for i in massive\_absol, а так же задавая блоки исключения, добавляем, либо умножаем i на ранее созданные <<result1>> и <<result2>>.
            \item
                Если результат умножения <<result2>> равен 1, то присваиваем значению стоковое выражение "None"
            \item
                Выводим на экран резльтаты.
        \end{enumerate}
    \end{algoritm}
        \texttt{Листинг Программы:}
    \begin{verbatim}
# -*- coding: utf-8 -*-
import random, math
massive,massive_absol=[],[]
n = int(input("Введите количество элементов массива: \n"))
result1=0
result2=1
for i in range(n):
    massive.append(random.randint(-5,5))
while True:
    y = int(input("Введите y (элемент массива в диапазоне от [-5:5]): \n"))
    if -5 <= y  and y <= 5:
        if y not in massive:
            massive.remove(massive[random.randrange(n)])
            massive.append(y)
            break
        else:
            break
for i in massive:
    massive_absol.append(int(math.fabs(i)))
for i in massive_absol:
    if i < y:
        result1+=i
for i in massive_absol:
    if i > y:
        result2*=i
if result2 == 1:
    result2 = "None"
# print(massive)
# print(massive_absol)
print("сумма элементов меньше y: " + str(result1))
print("умножение элементов больше y: " + str(result2))
    \end{verbatim}
    \begin{center}
        Результат работы программы:
    \end{center}
    \begin{verbatim}
Введите количество элементов массива: 
12
Введите y (элемент массива в диапазоне от [-5:5]): 
4
сумма элементов меньше y: 6
умножение элементов больше y: 25
    \end{verbatim}
    \end{lab4}
    
\end{document}
